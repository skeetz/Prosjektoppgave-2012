\documentclass[11pt, oneside, a4paper]{article}  

% Input and math
\usepackage[utf8]{inputenc}
\usepackage{amsmath,amssymb,amsfonts}
\usepackage{amsthm}

% Hyperlinks
\usepackage{hyperref}

% Colors
\usepackage{color}
\definecolor{dkgreen}{rgb}{0,0.6,0}
\definecolor{gray}{rgb}{0.5,0.5,0.5}


% Source code listings (see below begindoc), graphics
\usepackage{listings}
\usepackage{graphicx}
\usepackage{subfig}

\begin{document}

% For source code listings
\lstset{language=Matlab,
   keywords={break,case,catch,continue,else,elseif,end,for,function,
      global,if,otherwise,persistent,return,switch,try,while},
   basicstyle=\ttfamily,
   keywordstyle=\color{blue},
   commentstyle=\color{red},
   stringstyle=\color{dkgreen},
   numbers=left,
   numberstyle=\tiny\color{gray},
   stepnumber=1,
   numbersep=10pt,
   backgroundcolor=\color{white},
   tabsize=4,
   showspaces=false,
   showstringspaces=false}

\title{Blind Source Separation\\--\\Literature Search Protocol}
\author{Anders Røsæg Pedersen\\Ulf Nore}
\date{NTNU, Fall 2012}    % type date between braces
\maketitle

\begin{abstract}
Blabla..
\end{abstract}

\tableofcontents

\section{Introduction}

The blind source separation problem refers to the process of recovering one or more signals that have been mixed in some unknown manner and possibly also contamined by noise. Without any assumptions on the mixing process, this problem is ill-poised. In practice therefore, all BSS methods rely on some stylized fact about the nature of the signals and/or the mixing process. It is therefore useful to dichotomize BSS methods by these assumptions.

Arguably, two of the most important facts characterizing a mixing process, are its temporal dynamics and the number of degrees of freedom. The first point refers to whether the nature of the mixing process changes over time, that is if the mixing matrix at time $t+k$ is different from that at time $t$ for $k>0$. The number of degrees of freedom is the same concept as in linear algebra - the connection is apparent by seeing the mixing process as a system of linear equations. If $m$ is the number of observed signals and $n$ the number of sources, the system is said to be \emph{underdetermined} if $m<n$ and conversely \emph{overdetermined} if $m>n$. 

We can also differentiate between method based on the nature of input data. Early BSS research often considered the case of $n=m$, which allows one to work with data in the time domain. For undetermined systems, it is commonplace to work with some transformation of the data, which in the case of audio data a time-frequency representation. Common methods include the \emph{short-term Fourier transform} and the \emph{wavelet transform}.

\section{Literature Review Process} % Anders



\section{Literature Overview}


\subsection{Independent Component Analysis} %% Ulf

%% Standard ICA
Among the most common approaches to blind source separation is
independent component analysis. Presented first by \ref{comon94}, the
classical approach has been reinterpreted in different ways, as well
as extended to handle more general cases. 

Common definitions of ICA use either the maximization of independence
or minimization of mutual information between the source
signals. Computationally, ICA can be stated in many different forms,
hereunder neural networks and  maximum likelihood.



Bell+ andre folk: An information-maximization approach, Fast Blind
source separation based on information theory
Hyvarinen: Independent component analysis: algorithms and applications

%% Extensions to Single Channel Mix
Bach Blind one-microphone speech separation a spectral learning approach


\subsection{Factoral Hidden Markov Models} %% Anders


\subsection{Wavelets} %Ulf
Ichir Hidden markov models

\section{Conclusion}

\theappendix

\section{Protocol}


\begin{thebibliography}{99}
\bibitem{comon94} Comon, Pierre (1994): "Independent Component Analysis: a new concept?", Signal Processing, 36(3):287–314

  
\end{thebibliography}




\end{document}
