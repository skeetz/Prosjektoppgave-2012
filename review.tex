\documentclass[11pt, oneside, a4paper]{article}  

% Input and math
\usepackage[utf8]{inputenc}
\usepackage{amsmath,amssymb,amsfonts}
\usepackage{amsthm}

% Hyperlinks
\usepackage{hyperref}

% Colors
\usepackage{color}
\definecolor{dkgreen}{rgb}{0,0.6,0}
\definecolor{gray}{rgb}{0.5,0.5,0.5}


% Source code listings (see below begindoc), graphics
\usepackage{listings}
\usepackage{graphicx}
\usepackage{subfig}

\begin{document}

% For source code listings
\lstset{language=Matlab,
   keywords={break,case,catch,continue,else,elseif,end,for,function,
      global,if,otherwise,persistent,return,switch,try,while},
   basicstyle=\ttfamily,
   keywordstyle=\color{blue},
   commentstyle=\color{red},
   stringstyle=\color{dkgreen},
   numbers=left,
   numberstyle=\tiny\color{gray},
   stepnumber=1,
   numbersep=10pt,
   backgroundcolor=\color{white},
   tabsize=4,
   showspaces=false,
   showstringspaces=false}

\title{Blind Source Separation\\--\\Literature Search Protocol}
\author{Anders Røsæg Pedersen\\Ulf Nore}
\date{NTNU, Fall 2012}    % type date between braces
\maketitle

%% \begin{abstract}
%% \end{abstract}

\tableofcontents



\section{Research Agenda}
To systematically review current technology for blind source separation (BSS), with particular emphasis on the particular subproblem of single channel blind source separation (SCBSS).

The blind source separation problem consists transforming a set of observed signals that has undergone some particular mixing process back to the original unobserved signals. The ``blind'' part of the problem refers to the fact that the nature of the mixing process is unknown. 

\section{Background}



\subsection*{Research Question}
\begin{enumerate}
  \item What are the state-of-the-art approaches to blind source separation; with particular emphasis on single channel BSS.
  \item Investigate the properties of the techniques identified in Question 1, what assumptions do they make about the nature of the sources and the mixing process?
  \item What empirical evidence is there to document the performance of the techniques identified in Question 1?
\end{enumerate}



\section{Search Strategy}

\subsection{Databases}

\subsection{Search Terms}


\begin{itemize}
  \item blind source separation
  \item single channel blind source separation
  \item single mixture blind source separation
  \item single microphone blind source separation
\end{itemize}


\subsection{Inclusion and Quality Criteria}


\end{document}
